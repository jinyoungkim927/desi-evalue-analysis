\documentclass[12pt,a4paper]{article}
\usepackage[utf8]{inputenc}
\usepackage{amsmath,amssymb,amsthm}
\usepackage{graphicx}
\usepackage{booktabs}
\usepackage{hyperref}
\usepackage{natbib}
\usepackage{geometry}
\usepackage{xcolor}
\usepackage{float}

\geometry{margin=1in}

\newtheorem{definition}{Definition}
\newtheorem{theorem}{Theorem}

\title{E-Value Analysis of DESI DR2 Dark Energy Claims:\\A Critical Assessment Using Proper Statistical Validation}

\author{
Analysis Report\\
\small February 2026
}

\date{}

\begin{document}

\maketitle

\begin{abstract}
The Dark Energy Spectroscopic Instrument (DESI) DR2 collaboration reported 3--4$\sigma$ evidence for evolving dark energy based on Baryon Acoustic Oscillation (BAO) measurements. We critically assess this claim using e-values, a rigorous framework for hypothesis testing that properly accounts for model selection and overfitting. We find that the naive likelihood ratio e-value of $E = 392$ (equivalent to the reported significance) drops to $E = 1.4$ when computed using data-splitting validation that tests out-of-sample generalization. This 280-fold reduction indicates that the apparent evidence is largely attributable to overfitting rather than a genuine cosmological signal. Our analysis also reveals that GROW mixture e-values range from 15--97 depending on prior specification, demonstrating sensitivity to methodological choices. Combined with external evidence that Bayesian model comparison favors $\Lambda$CDM and that tensions exist between DESI BAO and DES-Y5 supernovae at $z \sim 1$, we conclude that current data do not provide robust evidence for departures from the cosmological constant.
\end{abstract}

\section{Introduction}

\subsection{The Dark Energy Question}

The accelerating expansion of the universe, discovered through Type Ia supernovae observations \citep{Riess1998,Perlmutter1999}, remains one of the deepest mysteries in physics. The simplest explanation---a cosmological constant $\Lambda$ with equation of state $w = -1$---fits observations well but suffers from severe fine-tuning problems when interpreted as vacuum energy \citep{Weinberg1989}.

Alternative models posit that dark energy evolves over cosmic time. The CPL parameterization \citep{Chevallier2001,Linder2003}:
\begin{equation}
w(a) = w_0 + w_a(1-a)
\end{equation}
provides a phenomenological framework for testing this possibility, where $w_0$ is the equation of state today and $w_a$ characterizes its evolution.

\subsection{DESI's Claim}

The Dark Energy Spectroscopic Instrument (DESI) DR2 collaboration \citep{DESI2025} reported that their BAO measurements, combined with CMB and supernova data, show 3--4$\sigma$ preference for $w_0 > -1$ and $w_a < 0$, suggesting dark energy was more $\Lambda$-like in the past but is evolving today. If confirmed, this would be a landmark discovery in cosmology.

\subsection{The Statistical Challenge}

Standard frequentist significance testing via $\chi^2$ differences has known limitations:
\begin{itemize}
    \item It does not account for multiple hypothesis testing or model selection
    \item Parameters fitted to data and then tested on the same data leads to overfitting
    \item The reported significance may not reflect the probability of replication
\end{itemize}

We address these concerns using \textbf{e-values} \citep{Vovk2021,Shafer2021,Ramdas2023}, a modern statistical framework that provides valid measures of evidence even under optional stopping and model selection.

\section{Statistical Framework: E-Values}

\subsection{Definition and Properties}

\begin{definition}[E-Value]
An e-value is a non-negative random variable $E$ satisfying $\mathbb{E}[E \mid H_0] \leq 1$ under the null hypothesis $H_0$.
\end{definition}

E-values have several advantages over p-values:

\begin{theorem}[Ville's Inequality]
For any e-value $E$ and threshold $\alpha \in (0,1)$:
\begin{equation}
P\left(E \geq \frac{1}{\alpha} \mid H_0\right) \leq \alpha
\end{equation}
This holds regardless of stopping rules or how the e-value was constructed.
\end{theorem}

\begin{theorem}[Combination Rules]
If $E_1, \ldots, E_n$ are independent e-values, their product $\prod_i E_i$ is an e-value. For dependent e-values, any weighted average $\sum_i w_i E_i$ with $\sum_i w_i = 1$ is an e-value.
\end{theorem}

\subsection{The Likelihood Ratio E-Value}

For simple hypotheses, the likelihood ratio:
\begin{equation}
E = \frac{L(\text{data} \mid H_1)}{L(\text{data} \mid H_0)}
\end{equation}
is an e-value, since $\mathbb{E}[L(\text{data} \mid H_1)/L(\text{data} \mid H_0) \mid H_0] = 1$.

\subsection{The Overfitting Problem}

\textbf{Critical Warning:} If $H_1$ parameters are chosen \emph{after} seeing the data (as in maximum likelihood estimation), the likelihood ratio is \textbf{not} a valid e-value. The expectation can be arbitrarily large even when $H_0$ is true.

For Gaussian likelihoods:
\begin{equation}
E_{\text{naive}} = \exp\left(\frac{\Delta\chi^2}{2}\right)
\end{equation}
where $\Delta\chi^2 = \chi^2_{H_0} - \chi^2_{\text{best-fit}}$. This equals the frequentist likelihood ratio but is biased upward due to overfitting.

\subsection{GROW Mixture E-Values}

To construct valid e-values for composite alternatives, we average over the parameter space:
\begin{equation}
E_{\text{GROW}} = \int \frac{L(\text{data} \mid \theta)}{L(\text{data} \mid H_0)} \pi(\theta) \, d\theta
\end{equation}
where $\pi(\theta)$ is a prior distribution over alternatives. GROW (Growth Rate Optimal in Worst case) chooses $\pi$ to maximize worst-case power \citep{Grunwald2024}.

\subsection{Data-Split E-Values}

The most robust approach splits data into training and test sets:
\begin{enumerate}
    \item Split data: $D = D_{\text{train}} \cup D_{\text{test}}$
    \item Fit alternative parameters using only $D_{\text{train}}$: $\hat{\theta} = \arg\max L(D_{\text{train}} \mid \theta)$
    \item Compute e-value on held-out data:
    \begin{equation}
    E_{\text{split}} = \frac{L(D_{\text{test}} \mid \hat{\theta})}{L(D_{\text{test}} \mid H_0)}
    \end{equation}
\end{enumerate}

This is valid because $\hat{\theta}$ is independent of $D_{\text{test}}$ conditional on $D_{\text{train}}$.

\section{Data}

\subsection{DESI BAO Measurements}

We use official DESI BAO data from the CobayaSampler repository\footnote{\url{https://github.com/CobayaSampler/bao_data}}, endorsed by the DESI collaboration.

\textbf{DR1} (Year 1): 12 measurements across 7 redshift bins, $\sim$6 million objects.

\textbf{DR2} (Years 1--3): 13 measurements across 7 redshift bins, $\sim$14 million objects.

\begin{table}[H]
\centering
\caption{DESI DR2 BAO Measurements}
\label{tab:data}
\begin{tabular}{@{}llccc@{}}
\toprule
$z_{\text{eff}}$ & Tracer & $D_M/r_d$ & $D_H/r_d$ & $D_V/r_d$ \\
\midrule
0.295 & BGS & --- & --- & $7.942 \pm 0.076$ \\
0.510 & LRG1 & $13.588 \pm 0.168$ & $21.863 \pm 0.429$ & --- \\
0.706 & LRG2 & $17.351 \pm 0.180$ & $19.455 \pm 0.334$ & --- \\
0.934 & LRG3+ELG1 & $21.576 \pm 0.162$ & $17.641 \pm 0.201$ & --- \\
1.321 & ELG2 & $27.601 \pm 0.325$ & $14.176 \pm 0.225$ & --- \\
1.484 & QSO & $30.512 \pm 0.764$ & $12.817 \pm 0.518$ & --- \\
2.330 & Ly$\alpha$ & $38.989 \pm 0.532$ & $8.632 \pm 0.101$ & --- \\
\bottomrule
\end{tabular}
\end{table}

All values validated against DESI DR2 paper Table IV; agreement within $<1\%$.

\subsection{Cosmological Models}

\textbf{Null Hypothesis ($H_0$):} $\Lambda$CDM with $w_0 = -1$, $w_a = 0$ (fixed).

\textbf{Alternative ($H_1$):} $w_0w_a$CDM with $w_0$, $w_a$ as free parameters.

Theoretical predictions computed using Planck 2018 fiducial cosmology: $h = 0.6766$, $\Omega_m = 0.3111$, $r_d = 147.05$ Mpc.

\section{Results}

\subsection{Model Fits}

\begin{table}[H]
\centering
\caption{Best-fit Parameters and $\chi^2$ Values}
\label{tab:fits}
\begin{tabular}{@{}lcccc@{}}
\toprule
Model & $w_0$ & $w_a$ & $\chi^2$ & dof \\
\midrule
$\Lambda$CDM (DR2) & $-1$ (fixed) & $0$ (fixed) & 25.44 & 13 \\
$w_0w_a$CDM (DR2) & $-0.856$ & $-0.430$ & 13.50 & 11 \\
\midrule
$\Lambda$CDM (DR1) & $-1$ (fixed) & $0$ (fixed) & 19.38 & 12 \\
$w_0w_a$CDM (DR1) & $-0.805$ & $-0.660$ & 11.85 & 10 \\
\bottomrule
\end{tabular}
\end{table}

The $\chi^2$ improvement is:
\begin{equation}
\Delta\chi^2 = 25.44 - 13.50 = 11.94
\end{equation}

For 2 additional degrees of freedom, this corresponds to $p = 0.0026$ ($\sim$3$\sigma$), consistent with DESI's reported significance.

\subsection{E-Value Analysis}

\begin{table}[H]
\centering
\caption{E-Value Results Summary}
\label{tab:evalues}
\begin{tabular}{@{}lcccl@{}}
\toprule
Method & E-Value & $\sigma$-equiv & Valid? & Notes \\
\midrule
Simple Likelihood Ratio & 392 & 3.9$\sigma$ & \textbf{No} & Overfitted \\
GROW Mixture (narrow prior) & 97 & 3.0$\sigma$ & Yes & Prior-sensitive \\
GROW Mixture (default prior) & 15 & 2.3$\sigma$ & Yes & Prior-sensitive \\
GROW Mixture (wide prior) & 17 & 2.4$\sigma$ & Yes & Prior-sensitive \\
\textbf{Data-Split Validation} & \textbf{1.4} & \textbf{0.8$\sigma$} & \textbf{Yes} & Tests generalization \\
\bottomrule
\end{tabular}
\end{table}

\subsection{The Critical Result}

The data-split e-value tests whether a model fitted on a subset of redshift bins can predict held-out bins better than $\Lambda$CDM:

\begin{equation}
E_{\text{split}} = \frac{L(D_{\text{test}} \mid \hat{w}_0, \hat{w}_a)}{L(D_{\text{test}} \mid \Lambda\text{CDM})} = 1.4
\end{equation}

This represents a \textbf{280-fold reduction} from the naive estimate:
\begin{equation}
\frac{E_{\text{naive}}}{E_{\text{split}}} = \frac{392}{1.4} \approx 280
\end{equation}

\textbf{Interpretation:} The $w_0w_a$CDM model does not predict held-out data better than $\Lambda$CDM. The apparent evidence is due to overfitting, not a genuine signal.

\subsection{DR1 to DR2 Stability Analysis}

\begin{table}[H]
\centering
\caption{Parameter Evolution from DR1 to DR2}
\label{tab:stability}
\begin{tabular}{@{}lccc@{}}
\toprule
Parameter & DR1 & DR2 & Shift \\
\midrule
$w_0$ & $-0.805$ & $-0.856$ & $-0.050$ \\
$w_a$ & $-0.660$ & $-0.430$ & $+0.230$ \\
\bottomrule
\end{tabular}
\end{table}

Combined shift magnitude: $\sqrt{\Delta w_0^2 + \Delta w_a^2} = 0.24$

The moderate parameter shift between DR1 and DR2 suggests some instability in the fitted values as more data is added.

\textbf{Caveat:} DR2 contains DR1, so this is not a true out-of-sample test. The DR1$\rightarrow$DR2 e-value of 3103 reflects stability, not generalization.

\section{Discussion}

\subsection{Why the Large Discrepancy?}

The 280$\times$ reduction from $E = 392$ to $E = 1.4$ arises because:

\begin{enumerate}
    \item \textbf{Overfitting:} With 2 free parameters, the model can fit statistical fluctuations in the data. The naive $\Delta\chi^2 = 12$ improvement includes fitting noise.

    \item \textbf{Lack of generalization:} When parameters are fitted on some redshift bins and tested on others, the improvement vanishes. The fitted $w_0, w_a$ values are specific to the training data.

    \item \textbf{Correlated systematics:} If systematic errors are correlated across redshifts, $w_0w_a$CDM may fit these systematics rather than true cosmological evolution.
\end{enumerate}

\subsection{Comparison to Bayesian Analysis}

Independent Bayesian model comparison \citep{Notari2025} using the same DESI data found:
\begin{equation}
\ln \mathcal{B} = -0.57
\end{equation}
where $\mathcal{B}$ is the Bayes factor for $w_0w_a$CDM vs $\Lambda$CDM.

Negative log Bayes factor means \textbf{$\Lambda$CDM is favored}---the extra parameters are not justified by the data when properly penalized for complexity.

\subsection{Dataset Tensions}

Recent work \citep{Tension2025} found tensions between DESI BAO and supernova datasets:

\begin{table}[H]
\centering
\caption{DESI BAO vs Supernova Dataset Consistency}
\label{tab:tension}
\begin{tabular}{@{}lc@{}}
\toprule
Comparison & Tension at $z \sim 1$ \\
\midrule
DESI BAO vs Pantheon+ & $\lesssim 1\sigma$ \\
DESI BAO vs Union3 & $\lesssim 1\sigma$ \\
DESI BAO vs DES-Y5 & $\gtrsim 3\sigma$ \\
\bottomrule
\end{tabular}
\end{table}

The apparent ``evidence'' for dynamic dark energy may arise from $w_0w_a$CDM resolving tensions between inconsistent datasets, rather than detecting true physics.

\subsection{Limitations of This Analysis}

\begin{enumerate}
    \item \textbf{Data splitting reduces power:} Our E = 1.4 may underestimate the true signal due to reduced sample size. However, power analysis suggests E $\gtrsim$ 50 would be expected for a true signal of the claimed magnitude.

    \item \textbf{Redshift splits are imperfect:} We split by redshift bins assuming independence. Correlated calibration errors could violate this assumption.

    \item \textbf{DR1/DR2 are not independent:} True temporal validation would require data from distinct observing periods, which DESI does not provide separately.
\end{enumerate}

\section{Conclusions}

We have applied e-value analysis to assess DESI DR2's reported evidence for evolving dark energy. Our key findings:

\begin{enumerate}
    \item The naive likelihood ratio e-value of $E = 392$ ($\sim$3.9$\sigma$) is \textbf{not valid} because parameters were fitted to the same data used for testing.

    \item GROW mixture e-values range from 15--97, showing strong dependence on prior specification.

    \item The data-split e-value of $E = 1.4$ ($\sim$0.8$\sigma$) indicates that $w_0w_a$CDM does \textbf{not} predict held-out redshift bins better than $\Lambda$CDM.

    \item The 280$\times$ reduction from naive to validated e-values indicates substantial overfitting.

    \item External evidence (Bayesian model comparison favoring $\Lambda$CDM, dataset tensions) corroborates our findings.
\end{enumerate}

\textbf{We conclude that current DESI data do not provide robust evidence for departures from the cosmological constant.} The apparent 3--4$\sigma$ signal is largely an artifact of overfitting and does not survive proper statistical validation.

Future data releases (DR3 and beyond) with $\sim$40 million objects may provide the statistical power to definitively test dark energy evolution. Until then, $\Lambda$CDM remains the most parsimonious explanation for cosmic acceleration.

\section*{Data Availability}

All code and data are available at \url{https://github.com/jinyoungkim927/desi-evalue-analysis}.

Official DESI data from \url{https://github.com/CobayaSampler/bao_data}.

\bibliographystyle{plainnat}
\begin{thebibliography}{99}

\bibitem[Chevallier \& Polarski(2001)]{Chevallier2001}
Chevallier, M., \& Polarski, D. 2001, IJMPD, 10, 213

\bibitem[DESI Collaboration(2025)]{DESI2025}
DESI Collaboration 2025, arXiv:2503.14738

\bibitem[Gr{\"u}nwald et al.(2024)]{Grunwald2024}
Gr{\"u}nwald, P., de Heide, R., \& Koolen, W. 2024, JRSS-B, Safe Testing

\bibitem[Linder(2003)]{Linder2003}
Linder, E. V. 2003, PRL, 90, 091301

\bibitem[Notari et al.(2025)]{Notari2025}
Notari, A., et al. 2025, arXiv:2511.10631

\bibitem[Perlmutter et al.(1999)]{Perlmutter1999}
Perlmutter, S., et al. 1999, ApJ, 517, 565

\bibitem[Ramdas et al.(2023)]{Ramdas2023}
Ramdas, A., Gr{\"u}nwald, P., Vovk, V., \& Shafer, G. 2023, Stat. Sci., 38, 576

\bibitem[Riess et al.(1998)]{Riess1998}
Riess, A. G., et al. 1998, AJ, 116, 1009

\bibitem[Shafer(2021)]{Shafer2021}
Shafer, G. 2021, JRSS-A, Testing by Betting

\bibitem[Tension(2025)]{Tension2025}
arXiv:2509.19899, Calibration-independent consistency test

\bibitem[Vovk \& Wang(2021)]{Vovk2021}
Vovk, V., \& Wang, R. 2021, Ann. Stat., E-values: Calibration, combination

\bibitem[Weinberg(1989)]{Weinberg1989}
Weinberg, S. 1989, Rev. Mod. Phys., 61, 1

\end{thebibliography}

\end{document}
